%%=============================================================================
%% Methodologie
%%=============================================================================

\chapter{\IfLanguageName{dutch}{Methodologie}{Methodology}}
\label{ch:methodologie}

%% TODO: Hoe ben je te werk gegaan? Verdeel je onderzoek in grote fasen, en
%% licht in elke fase toe welke stappen je gevolgd hebt. Verantwoord waarom je
%% op deze manier te werk gegaan bent. Je moet kunnen aantonen dat je de best
%% mogelijke manier toegepast hebt om een antwoord te vinden op de
%% onderzoeksvraag.
To be able to measure the differences in performance, three Angular applications will be developed. All applications will recieve SignalR updates from the server and draw several real-time charts in the browser and animate the updated data. The first application will let Zone.js and default Angular change detection handle everything. In the second application the changeDetectorRef will be injected and the onPush strategy will be used to achieve the same goal. In the third application, Zone.js will be excluded from the project and the Angular Ivy internal private API will be used to handle change detection fully manual. Afterwards the performance of all three applications can be measured and compared.

\section{Server side}
To test the performance of rendering charts in Angular based on real-time data, an API has to be developed. In this experiment a .NET Core Web API will be developed that has two endpoints. The first endpoint is to deliver the chart settings to the front-end. This way, the client can make an HTTP call to the server to know how many charts of which type have to be rendered. The second responsability of the API is to function as a SignalR hub that sends simulated data to the client to update the charts with a high frequency.

\section{Default application}
When de server side application is finished to provide the data, the first client application will be developed and can provide a base solution for the following two. The logic of the dashboard will be isolated into a feature module, which will be imported by the app module. This dashboard module will consist of a service for fetching the configurations of the charts from the API and a service for handling the signalR connection. Further, some models will be added that hold the settings of a chart and an enum for the type. To draw the charts, the charting library ng2-charts will be installed together with chart.js. Four types of charts will be implemented in the app and each type will have its own component inside the dashboard module. Furthermore a base chart component can hold some logic that is similar across the chart components.

\section{OnPush application}
For the onPush application, the default application can serve as a base solution to start with. The onPush application will be created as a copy from the first one. Then the change detection strategy of the chart components will be set to onPush and change detection has to be triggered manually when the SignalR service provides a new value.

\section{Zoneless application}
The third and last application will also be created with a copy from the first one. Afterwards Zone.js can be excluded from the project and Angular has to be configured to work without zones. When this is done succesfully, the manual triggering of the change detection cycle has to be implemented where a value that is reflected in the view changes.